\chapter{Problem Statement}\label{ch:probstate}
 %************************************************
Although numerous governmental and industry reports\footnote{\citep{RDG2017},\citep{DepartmentforTransport2011},\citep{TechnicalStrategyLeadershipGroup2012b}} have espoused improved data integration as a driver of reduced costs for many rail industry stakeholders and an improved travel experience for passengers, efforts towards this goal have been slow to implement. Costs would be reduced for both infrastructure and operating companies by enabling the implementation of other technologies, such as predictive maintenance which reduces their direct operating costs, along with a reduction of the cost of alterations or extensions to information systems. The benefits of predictive maintenance as discussed by \citet{QRE:QRE1634} and \citet{RDG2017}, include reduced costs by reducing unnecessary replacement of working equipment as part of planned maintenance and expensive failures caused by inadequate preventative maintenance. 

As discussed in a study commissioned by the American National Institute of Standards and Technology and carried out by \citep{Gallaher2004} the capital facilities industry (Large scale construction) in the USA projects that it could save \$15.8~billion were it to adopt ontology for data integration. Other domains have more mature implementations of ontology for data integration, most notably biomedical research where such technology is not a matter of research, but everyday use. In the consumer domain, the popular \enquote{Siri} virtual personal assistant makes heavy use of ontology for knowledge representation and question answering.

Previous studies such as \citep{Verstichel2011a}, \citep{Tutcher2013}, and \citep{Morris} have shown ontology to be a useful tool for data integration in many domains, including rail, and it is this integration which makes other technologies possible. Once the range of heterogeneous datasources that many industries have, or have had, are modelled as an ontology the data contained there in can be combined and is made accessible throughout the domain. Additional benefits are possible if the rules describing how decisions are made in the domain are also encoded in the ontology, enabling better decision making and more oversight from domain experts. Ontologies already exist for the rail domain, many of which were created by the same studies as found that there would be a benefit from using ontology in the domain. The work reported by \citet{Tutcher2015}, included the design of a linked set of ontologies for the rail domain. Previously ontologies were also built as part of the InteGrail project, reported by \citet{Kopf2010}, and more recently as part of the Trapist project reported by \citet{Bhatti2016}. Additionally there has been work done in the commercial sector; notably ERTMS solutions of Brussels\footnote{\url{https://www.ertmssolutions.com/}} and Televic Rail of Izegem\footnote{\url{http://www.televic-rail.com/en/}} have done work in this area.

It can be seen that the market is starting to respond to the industry need, however there is still no significant uptake of ontologies or linked data in the rail domain. There are demonstrators, such as those reported in: \citep{Bhatti2016}, \citep{Tutcher2013} or earlier in \citep{Kopf2010}, but, in contrast to other sectors, commercial uptake remains limited. The information environment in the rail domain is very diverse and this may have impeded uptake of ontologies, as such the extent to which this is a barrier to uptake requires investigation.
In particular whilst many studies have produced data models or demonstrators there are no national scale implementations, thus investigating whether this is possible at reasonable cost would be beneficial. 

The transition from the current situation, that of many incompatible heterogeneous datasources to a system where queries can seamlessly retrieve data from multiple sources will be a complex process. It has been established by previous studies that ontology will make that possible, but the transition has not been studied in depth. Work has been done, both academically and commercially, to allow the use of relational databases with linked data and ontology. Whilst relational datasources are straight forward to convert completely unstructured data, such as technical drawings, sensor data streams, or flat text files will require further investigation. Given that tools exist to make the transition for relational data sources then it would be useful to ascertain whether it is possible to make similar tools for other data sources.

Once the transition to using ontology and linked data has been made, or even begun, another challenge must be faced, that of the skills gap in the software engineering domain centred on ontology engineering, which presents a barrier to uptake of ontology in all domains, including rail. In moving from the theoretical phase, technology readiness level 4 or 5, to implementation there is a need for both software engineers who can work with ontology datastores and ontology engineers who can construct domain models. In the long term this gap can be filled with education, however, given that neither linked data nor ontology are currently included in the syllabuses of most university level computer science courses this is not a short term solution. We should then consider whether tools could be created to help plug that gap.

If implemented fully, an ontology (or a linked set of ontologies) would hold all the logic and decision making rules used in any new software, leaving only interfaces (with humans or external equipment) to the software developer.  When creating a new interface is required, for example display on a new piece of hardware, or when a new sensor is attached to the system it would be a simple software engineering task. This would be accomplished by providing the software developer with a webservice to call which would handle the operation, thus separating the roll of ontology engineer from the roll of a software developer. By removing the specialist tasks from more generalist software engineers, the development of new systems, or modification of old, to incorporate ontologies for data storage is made possible. This eases data integration and enables all the benefits available to the rail domain discussed in \autoref{ch:litreview}.

After ontology has been applied to the rail domain another challenge to face is that of high velocity and volume data. When deployed on a national scale some data, such as that from sensors or cameras simply arrives too fast and in too large of a quantity to express as triples and store in an ontology. Such data needs to be stored separately, however more value would be available to the domain were it stored in the ontology as such, in line with the proposals in \citep{Tutcher2015}, a compromise solution is possible whereby the fine grained data resides in a suitable store and a link, along with a summary resides in the triple store. The aggregation of these storage media is a task that will need to be carried out where ever high volume sensor data is used, which is a common occurrence in the rail-domain, thus it is reasonable to ask if this could be done once to avoid unnecessary repetition.

Another problem that will be faced after ontology is adopted is that of changing interfaces to triple stores and potentially as the market evolves even changing triple stores. It would be problematic if as a new version of a triple store was released it broke existing industry software. Whilst vendors will naturally work together with industry clients to minimise this it is regrettably the case that interfaces do change with time. It is also the case that as the market matures different triple stores may present them selves as the most appropriate back-end and it would be beneficial to the industry if migration was possible. It would there for be good to investigate if it is possible to isolate the rail industry from changes to the triple stores.

The last issue this thesis will seek to investigate is that of information security. As shown in the literature review this issue is under broader consideration in the rail domain, however the question of how to secure datastores with no inbuilt security remains outstanding and is related to the question of datastore aggregation.

The questions may then be summarised thus:
\begin{itemize}
	\item \QuestionOtherData
	\item \QuestionSkillz 	
	\item \QuestionCombine
	\item \QuestionChange
	\item \QuestionCanOntologyScale
	\item \QuestionSecurity		
\end{itemize}

The remainder of this document is devoted to answering the questions above. 



\noindent
\begin{tabularx}{\textwidth}{@{}Xl@{}} 
\textsc{Question} & \textsc{Investigated In}\\ 
\arrayrulecolor{LightSteelBlue}\midrule[\heavyrulewidth]
\QuestionOtherData & \mycell{Chapter four \\ Chapter six}\\ \addlinespace
\QuestionSkillz    & \mycell{Chapter five \\ Chapter six}\\ \addlinespace
\QuestionCombine   & \mycell{Chapter five \\ Chapter six}\\ \addlinespace
\QuestionChange    & Chapter five \\                        \addlinespace
\QuestionCanOntologyScale & Chapter six \\                  \addlinespace
\QuestionSecurity  & \mycell{Chapter five \\ Chapter six} \\ 
\bottomrule
\end{tabularx}




