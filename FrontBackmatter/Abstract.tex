%*******************************************************
% Abstract
%*******************************************************
%\renewcommand{\abstractname}{Abstract}
\pdfbookmark[1]{Abstract}{Abstract}
\begingroup
\let\clearpage\relax
\let\cleardoublepage\relax
\let\cleardoublepage\relax

\chapter*{Abstract}
The exchange of information is crucial to the operation of railways; starting with the distribution of timetables, information must constantly be exchanged in any railway network. The slow evolution of the information environment within the rail industry has resulted in the existence of a diverse range of systems, only able to exchange information essential to railway operations. Were the cost of data integration reduced, then further cost reductions and improvements to customer service would follow as barriers to the adoption of other technologies are removed. 

The need for data integration has already been studied extensively and has been included in the UK industry's rail technical strategy however, despite it's identification as a key technique for improving integration, uptake of ontology remains limited. This thesis considers techniques to reduce barriers to the take up of ontology in the UK rail industry, and presents a case study in which these techniques are applied. Amongst the key barriers to uptake identified are a lack of software engineers with ontology experience, and the diverse information environment within the rail domain. Techniques to overcomes these barriers using software based tools are considered, and example tools produced which aid the overcoming of these barriers.

The case study presented is of a degraded mode signalling system, drawing data from a range of diverse sources, integrated using an ontology. Tools created to improve data integration are employed in this commercial project, successfully combing signalling data with (simulated) train positioning data.


\endgroup			

\vfill