%*******************************************************
% Abstract
%*******************************************************
%\renewcommand{\abstractname}{Abstract}
\pdfbookmark[1]{Abstract}{Abstract}
\begingroup
\let\clearpage\relax
\let\cleardoublepage\relax
\let\cleardoublepage\relax

\chapter*{Abstract}
The exchange of information is crucial to the operation of railways; starting with the distribution of timetables, information must constantly be exchanged in any railway network. The slow evolution of the information environment within the rail industry has resulted in the existence of a diverse range of systems, only able to exchange information essential to railway operations. Were the cost of data integration reduced, then further cost reductions would follow as barriers to the adoption of other technologies are removed. 

Using linked data and ontology to achieve integration has many advantages over other less information-rich and more prescriptive formats; for example by using a linked data architecture it is possible for anyone to make data available to the industry. Another key advantage of ontology and linked data over other formats for interchange is that it is possible to exchange not merely information, but the logic by which the industry operates. 

The need for data integration has already been studied extensively and has been included in the UK industry's rail technical strategy, however uptake of ontology remains limited. This thesis considers means of reducing barriers to the take up of ontology in the UK rail industry.

\endgroup			

\vfill